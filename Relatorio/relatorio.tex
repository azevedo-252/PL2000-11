\documentclass[12pt,a4paper,oneside]{article}
\usepackage[utf8]{inputenc}
\usepackage[portuges]{babel}
\usepackage[pdftex]{graphicx}
\usepackage{lscape}
\usepackage{listings}
\usepackage{indentfirst}
\newenvironment{codefold}{}{}

\title{\textbf{Processamento de Linguagens}\\Compilador para a Linguagem LogoLISS}
\author{54776 - Rafael Abreu\\54789 - Bruno Azevedo\\54821 - Vítor Costa}
\date{\today}


\begin{document}
\begin{codefold}

\maketitle
\newpage
\tableofcontents
\newpage

\section{Introdução}
\indent Na disciplina de Processamento de Linguagens foi-nos proposto a realização de um trabalho prático utilizando a ferramenta Yacc. Tem como objectivo a prática de 
manipulação de uma linguagem a conhecendo a sua gramática. Neste relatório está explicado todo o procedimento para o seu desenvolvimento e o resultado final.\\

\newpage
\section{Escolha do Enunciado}

\indent Neste segundo trabalho, apenas tínhamos um enunciado disponível, não havendo opções de escolha. Este enunciado propunha-nos a criação de um interpretador léxico
para a linguagem LogoLISS, e disponibilizava-nos a sua gramática na íntegra. Mais tarde, foi-nos disponibilizado um segundo enunciado, em que a linguagem que tratava era
bastante mais simples que o LogoLISS, mas, para equilibrar o nível de dificuldade, não nos disponibilizava o gramática.\\
\indent Optámos pelo primeiro enunciado pela simples razão de já estarmos dentro do problema e mais ambientados com ele do que com o enunciado alternativo.\\

\newpage
\section{Desenvolvimento do Programa}

\indent A primeira grande dificuldade com que nos deparamos (e que se foi revelando ocasionalmente durante o desenvolvimento do trabalho) foi: ``O que é o LogoLISS?
O que é que é suposto determinado comando fazer? Como é que vou conseguir guardar esta informação de modo a que a VM consiga aceder e trabalhar com bons resultados?''.
Foi este tipo de obstáculos que tivemos de superar, ambientando-nos no mundo do LogoLISS e da máquina virtual (VM), principalmente à sua parte gráfica (GVM).\\
\indent Para que conseguissemos enfrentar tal barreiras e ultrapassa-las com sucesso, decidimos implementar o seguinte:

  \subsection{O Ficheiro FLEX e o ficheiro YACC}

  \indent Para dar início ao trabalho prático, começamos por implementar a sua base: os ficheiro FLEX e YACC. De seguida, encontra-se o principal dessa base:\\

\begin{verbatim}
{PROGRAM}                 { return PROGRAM; }
{DECLARATIONS}            { return DECLARATIONS; }
{STATEMENTS}              { return STATEMENTS; }
"->"                      { return ARROW; }
{INTEGER}                 { return INTEGER; }
{BOOLEAN}                 { return BOOLEAN; }
{STRING}                  { return STRING; }
{ARRAY}                   { /*return ARRAY;*/ }
{SIZE}                    { /*return SIZE;*/ }
{TRUE}                    { yylval.stringvalue=(char*)strdup(yytext);
                            return TRUE; }
{FALSE}                   { yylval.stringvalue=(char*)strdup(yytext);
                            return FALSE; }
{FORWARD}                 { return FORWARD; }
{BACKWARD}                { return BACKWARD; }
{RRIGHT}                  { return RRIGHT; }
{RLEFT}                   { return RLEFT; }
{PEN}                     { return PEN; }
{UP}                      { return UP; }
{DOWN}                    { return DOWN; }
{GOTO}                    { return GOTO; }
{WHERE}                   { return WHERE; }
"||"                      { return OR; }
"&&"                      { return AND; }
"**"                      { return POW; }
"=="                      { return EQUAL; }
"!="                      { return DIF; }
"<"                       { return MINOR; }
">"                       { return MAJOR; }
"<="                      { return MINOREQUAL; }
">="                      { return MAJOREQUAL; }
{IN}                      { return IN; }
[=\-,;!?/()\[\]\{\}*+]    {return(yytext[0]);}
{SUCC}                    { yylval.stringvalue=(char*)strdup(yytext);
                            return SUCC; }
{PRED}                    { yylval.stringvalue=(char*)strdup(yytext);
                           return PRED; }
{SAY}                     { return SAY; }
{ASK}                     { return ASK; }
{IF}                      { return IF; }
{THEN}                    { return THEN; }
{ELSE}                    { return ELSE; }
{WHILE}                   { return WHILE; }

{NUMBER}                  { yylval.stringvalue = strdup(yytext);
                            return NUMBER; }
{IDENTIFIER}              { yylval.stringvalue = strdup(yytext);
                            return IDENTIFIER; }
{STR}                     { yylval.stringvalue = strdup(yytext);
                            return STR; }

[\ \n\t\r]+               {;}

<*>.|\n                   { fprintf(stderr,"ERRO: %d '%s'\n",yylineno, yytext);}
\end{verbatim}

\indent Aqui está representado como é que o nosso programa consegue corresponder determinado comando a uma determinada acção: caso encontre qualquer coia que esteja do
lado esquerdo, vai retornar o que está do lado direito. Esse retorno irá ser utilizado no ficheiro YACC:

\begin{verbatim}
%token PROGRAM DECLARATIONS STATEMENTS ARROW
%token INTEGER BOOLEAN STRING FORWARD BACKWARD
%token RRIGHT RLEFT
%token PEN UP DOWN GOTO WHERE OR AND POW EQUAL
%token DIF MINOREQUAL MAJOREQUAL IN SAY ASK IF
%token THEN ELSE WHILE
%token <stringvalue>TRUE FALSE IDENTIFIER 
%token UMBER STR SUCC PRED
\end{verbatim}

\indent O que aqui está representado é a passagem de um comando para o seu respectivo ``token''. Com esta transformação, conseguimos manipular o comando da forma que
quisermos.\\
\indent De seguida, apresentamos as estruturas de dados que utilizamos no nosso trabalho prático:

  \subsection{Estruturas de Dados}
  
  \indent Para armazenarmos a informação relevante à medida que o nosso compilador percorre o programa, criamos várias estruturas de dados que nos ajudam nessa tarefa.\\
  \indent De seguida, está uma pequena explicação de cada uma dessas estruturas de dados (ver código em anexo):


    \subsubsection{VarHashTable}

\indent A máquina virtual, sendo uma máquina de pilhas que vai acumulando valores de diversos tipos, precisa de um mecanismo para associar, em runtime, esses valores a um
tipo, um identificador e o respectivo endereço na pilha.\\
\indent Decidimos que tal mecanismo seria composto por uma tabela de hash que a cada identificador associa um nodo da tabela. Cada nodo contem a informação enunciada
acima, o próprio identificador, o tipo do valor ao qual o identificador está associado e a endereço na stack onde esse valor foi alocado.\\
\indent Esta tabela, permite, em runtime, verificar se determinado identificador existe, e se existir, obtém informação suficiente para poder aceder ao seu valor na stack.
Também poderá validar os valores de input verificando o tipo do identificador.\\

\indent As funções associadas à tabela de hash correspondem às funcionalidades básicas de inserção e procura:

\begin{description}
 \item [initHash()] inicializa a tabela de hash;
 \item [hash()] função de hash que cria uma chave dado um identificador;
 \item [searchVar(char* id)] procura a variável na tabela e retorna caso exista;
 \item [insertVar(char* id, int type, int address)] insere uma nova variável na tabela hash.
\end{description}


    \subsubsection{VarTipo}

\indent Estrutura que no decorrer do parsing, quando encontra uma variável, guarda o identificador, o valor e o tipo de dados.\\

    \subsubsection{ConstTipo}

\indent Estrutura que no decorrer do parsing, quando encontra uma constante, guarda o valor e o tipo. Tipo este que pode representar valor nulo.\\

    \subsubsection{ListaVars}

\indent Lista ligada que, no decorrer do parsing, na Declaração de Variáveis, guarda as variáveis por declaração. Esta lista ligada é posteriormente utilizada para
armazenar as variáveis na tabela de hash.\\

  \subsection{Expressions}

\indent Por uma questão de simplificação, uma vez que as ``Expressions'' são expressões com um valor associado, declaramos que estes estados nada mais são que inteiros,
que corresponde ao seu valor.\\
\indent Basicamente, percorre a sua árvore de derivação até aos comandos base (adição, multiplicação de factores, etc.), calcula o seu valor e retorna-o.\\
\indent Este estado é utilizado nos mais variáveis comandos LogoLISS, desde comandos da Tartaruga até uma simples atribuição.\\

  \subsection{Declarations}

\indent O Declarations: Variables percorre as declarações de variáveis e por cada declaração vai adicionando as variáveis a uma lista ligada de variáveis. Cada nodo da
lista ligada corresponde a uma estrutura de dados que guarda as informações de cada variável, o identificador, o valor e o tipo de dados.Chegando ao fim de uma
declaração, as variáveis contidas na lista ligada são armazenadas na tabela de hash, os respectivos valores são "empilhados" na stack e a lista ligada é reinicializada
para guardar novas variáveis.

  \subsection{Statements}

\indent O Assignment Statement, por cada Assignment encontrado, calcula o valor da expressão (Expression) e coloca no topo da pilha. De seguida, o endereço do variável
(Variable) é retornado da tabela de hash para que a atribuição do resultado da expressão à variável, mais especificamente, à posição de memória da stack associada à
variável seja possível.\\

  \subsection{Os Comandos da Turtle}

\indent Este era o principal objectivo do trabalho: fazer um programa em LogoLISS capaz de criar uma tartaruga e capaz de a manipular, deslocando-a em várias direcções e
diferentes sentidos.\\
\indent Estes comandos representam todos os comandos que podemos fazer com a tartaruga (que no nosso projeto significa uma circunferência com centro na posição
(300, 200), com raio de 25, desenhado num ecrã 600x800, direcionada para cima (direcção = up) e com o rasto activo (mode = 1)). Os comandos implementados foram:

    \subsubsection{Setp}

\indent Este comando subdivide-se em 2 comandos, o FORWARD e o BACKWARD. Em primeiro lugar, a partir do endereço das coordenadas, vai buscar a posição actual da tartaruga
e guarda-a na stack, testa qual é a direcção ("up", "right", "down" ou "left") dada há tartaruga. Dependendo da rotação soma ou subtrai o valor da "expression" na
respectiva cordenada (x,y) da tartaruga e acualiza a nova posição da tartaruga. Desenha o seu rasto atráves das antigas e novas coordenadas calculadas, caso solicitado,
e só por fim desenha a nova posição da tartaruga.\\

    \subsubsection{Rotate}

\indent Representa as duas rotações possíveis dadas há tartaruga dependendo da actual direcção que tem. Subdivide-se em dois comandos diferentes: o RRIGhT e o RLEFT. O que
eles fazem é o seguinte: dependendo da direcção actual da tartaruga e da rotação dada calcula-se e atualiza-se a nova direcção na tartaruga.\\
  
    \subsubsection{Mode}

\indent representa os dois modos de desenho do rastro, para representar o movimento (Step) da tartaruga:PEN UP e PENDOWN. O que fazem é alterar o ``mode'' para 0 ou 1
caso seja para desenhar ou nao o rastro da tartaruga.\\

    \subsubsection{Dialogue}

\indent Mostra os comandos de diálogo ("say" e "ask") que podemos aplicar à tartaruga: Say\_Statement, que escreve uma ``expression'', e Ask\_Statement, que escreve uma
string e guarda numa variável o que foi lido do teclado.\\
\indent Por sua vez, o Say\_Statement divide-se no comando SAY ( Expression ), que calcula a ``Expression'' e coloca-a na stack consoante o tipo (``INTEGER'', ``BOOLEAN''
e ``STRING'') da ``Expression'' e imprime a respectiva expression; o Ask\_Statement divide-se no comando ASK ( STR , Variable ), que, em caso de a ``Variable'' nao estar
declarada, dá um erro, senão guarda na stack e mostra a string (STR), depois lê uma string do teclado (concluída por um enter), arquiva esta string na heap e o endereço
na satck. Converte o valor lido do teclado consoante o tipo da ``Variable'' e guarda esse valor na respectiva váriavel.

    \subsubsection{Location}

\indent Mostra ou edita a localização da tartaruga. Subdivide-se em GOTO NUMBER , NUMNBER, que em primeiro lugar, a partir do endereço das coordenadas, vai buscar a
posição actual da tartaruga e guarda-a na stack para depois poder desenhar a linha de rasto. Em seguida, arquiva e actualiza os valores das novas coordenadas (NUMBER's)
da tartaruga. Desenha o rasto da tartaruga, atráves das antigas e novas coordenadas calculadas, caso solicitado e só por fim desenha a nova posição da tartaruga na
posição pretendida; subdivide-se também no WHERE ?, que em primeiro lugar, a partir do endereço das coordenadas, vai buscar a posição actual da tartaruga e imprime em
primeiro lugar a coordenada x e depois a coordenada y.



\newpage
\section{Conclusão}

\indent Ao iniciar esta conclusão, assumida pela equipa responsável por este relatório sobre a realização de um compilador gerando código máquina de stack vitual para a
linguagem "LogoLISS - A Toy Language" como uma reflexão crítica, torna-se importante diferenciar o processo de desenvolvimento do resultado final.\\
\indent Como processo de desenvolvimento consideramos que o grupo de trabalho teve muitas dificuldades em interpretar o enunciado por falta de conhecimento do grupo
sobre o funcionamento duma máquina de pilhas.\\
\indent Tirando o factores acima referido, consideramos que o processo de desenvolvimento decorreu de uma forma bastante boa e eficaz. Como produto final, este responde,
quase na totalidade, aquilo que foi pedido no enunciado pela interpretação que efectuamos.
\indent Infelizmente, não conseguimos implementar tudo o que deveriamos, tal como condições if, ciclos while, muito devido à pressão que o 3º e último (esperamos) ano da
nossa licenciatura nos proporciona.\\


\newpage
\appendix

\section{Anexos}

\begin{verbatim}
                                                                     
                                                                     
                                             
+++++++++++++++++   logo.l  +++++++++++++++++++++

%{
#include "structures.h"
#include "hashFunctions.h"
#include "y.tab.h"
#include <stdlib.h>
#include <stdio.h>
#include <string.h>

%}

%option yylineno

PROGRAM 	[Pp][Rr][Oo][Gg][Rr][Aa][Mm]
DECLARATIONS 	[Dd][Ee][Cc][Ll][Aa][Rr][Aa][Tt][Ii][Oo][Nn][Ss]
STATEMENTS 	[Ss][Tt][Aa][Tt][Ee][Mm][Ee][Nn][Tt][Ss]
SUCC 		[Ss][Uu][Cc][Cc]
PRED 		[Pp][Rr][Ee][Dd]
IF 			[Ii][Ff]
ELSE		[Ee][Ll][Ss][Ee]
THEN		[Tt][Hh][Ee][Nn]
WHILE		[Ww][Hh][Ii][Ll][Ee]
INTEGER		[Ii][Nn][Tt][Ee][Gg][Ee][Rr]
BOOLEAN		[Bb][Oo][Oo][Ll][Ee][Aa][Nn]
STRING		[Ss][Tt][Rr][Ii][Nn][Gg]
ARRAY		[Aa][Rr][Rr][Aa][Yy]
SIZE		[Ss][Ii][Zz][Ee]
TRUE		[T][R][U][E]
FALSE		[F][A][L][S][E]
FORWARD		[Ff][Oo][Rr][Ww][Aa][Rr][Dd]
BACKWARD	[Bb][Aa][Cc][Kk][Ww][Aa][Rr][Dd]
RRIGHT		[Rr][Rr][Ii][Gg][Hh][Tt]
RLEFT		[Rr][Ll][Ee][Ff][Tt]
PEN			[Pp][Ee][Nn]
UP			[Uu][Pp]
DOWN		[Dd][Oo][Ww][Nn]
GOTO		[Gg][Oo][Tt][Oo]
WHERE		[Ww][Hh][Ee][Rr][Ee]
SAY			[Ss][Aa][Yy]
ASK			[Aa][Ss][Kk]
IN			[Ii][Nn]

NUMBER 						[0-9]+
IDENTIFIER 					[a-zA-Z][a-zA-Z0-9]*
STR						\"([^"\n]|\\\")*\"

%%

{PROGRAM}					{ return PROGRAM; }
{DECLARATIONS}					{ return DECLARATIONS; }
{STATEMENTS}					{ return STATEMENTS; }
"->"						{ return ARROW; }
{INTEGER}					{ return INTEGER; }
{BOOLEAN}					{ return BOOLEAN; }
{STRING}					{ return STRING; }
{ARRAY}						{ /*return ARRAY;*/ }
{SIZE}						{ /*return SIZE;*/ }
{TRUE}						{ return TRUE; }
{FALSE}						{ return FALSE; }
{FORWARD}					{ return FORWARD; }
{BACKWARD}					{ return BACKWARD; }
{RRIGHT}					{ return RRIGHT; }
{RLEFT}						{ return RLEFT; }
{PEN}						{ return PEN; }
{UP}						{ return UP; }
{DOWN}						{ return DOWN; }
{GOTO}						{ return GOTO; }
{WHERE}						{ return WHERE; }
"||"						{ return OR; }
"&&"						{ return AND; }
"**"						{ return POW; }
"=="						{ return EQUAL; }
"!="						{ return DIF; }
"<"						{ return MINOR; }
">"						{ return MAJOR; }
"<="						{ return MINOREQUAL; }
">="						{ return MAJOREQUAL; }
{IN}						{ return IN; }
[=\-,;!?/()\[\]\{\}*+]				{return(yytext[0]);}
{SUCC}						{ return SUCC; }
{PRED}						{ return PRED; }
{SAY}						{ return SAY; }
{ASK}						{ return ASK; }
{IF}						{ return IF; }
{THEN}						{ return THEN; }
{ELSE}						{ return ELSE; }
{WHILE}						{ return WHILE; }

{NUMBER}					{ yylval.stringvalue = strdup(yytext); return NUMBER; }
{IDENTIFIER}					{ yylval.stringvalue = strdup(yytext); return IDENTIFIER; }
{STR}						{ yylval.stringvalue = strdup(yytext); return STR; }

[\ \n\t\r]+					{;}

<*>.|\n						{ fprintf(stderr,"ERRO: %d '%s'\n",yylineno, yytext);}

%%

int yywrap() {
 return 1;
}

/*int yyerror(char *s){
	fprintf(stderr,"ERRO: %d %s\n",yylineno, s);
	return 0;
}*/








+++++++++++++++++++++   logo.y  ++++++++++++++++++++++++

%{
	#include <stdlib.h>
	#include <stdio.h>
	#include <ctype.h>
	#include <string.h>
	#include "structures.h"
	#include "hashFunctions.h"

	int addressG = 0;
	int height, width, xpos, ypos, raio;
	Direccao direccao;
	int mode; // PEN UP

	extern char* yytext;
	extern int yylineno;	

	extern ListaVars *nodo;
	extern VarHashTable varHashTable;

%}
%error-verbose

%token PROGRAM DECLARATIONS STATEMENTS ARROW
%token INTEGER BOOLEAN STRING FORWARD BACKWARD
%token RRIGHT RLEFT
%token PEN UP DOWN GOTO WHERE OR AND POW EQUAL
%token DIF MINOREQUAL MAJOREQUAL IN SAY ASK IF
%token THEN ELSE WHILE
%token <stringvalue>TRUE FALSE IDENTIFIER 
%token UMBER STR SUCC PRED

%left MINOR MAJOR MINOREQUAL MAJOREQUAL EQUAL AND POW DIF OR
%left '+' '-'
%left '*' '/'


%type <stringvalue>Add_Op Mul_Op Rel_Op
%type <intvalue>Type SuccOrPred Factor Term Single_Expression  Expression
%type <intvalue>SuccPred /*Array_Acess*/
%type <varTipo>Var Variable
%type <constTipo>Constant Value_Var Inic_Var

%union{
	int intvalue;
	char* stringvalue;
	VarTipo varTipo;
	ConstTipo constTipo;
}



%start Liss

%%

/***************************Program**************/

Liss 			: PROGRAM IDENTIFIER '{' Body '}' {printf("STOP\n");}
			;
	
Body 			: DECLARATIONS 	{
						height = 100;
						width = 100;
						xpos = 300;
						ypos = 200;
						raio = 25;
						mode = 1; //PEN UP
						direccao = up;
						init();
					}Declarations

 			  STATEMENTS {/*printHash();*/} Statements
			;


/***************************Declarations**************/

Declarations		: Declaration
			| Declarations Declaration
			;
	
Declaration 		: Variable_Declaration
			;


/***********Declarations: Variables************/

Variable_Declaration 	: Vars ARROW Type ';' 	{saveVars($3);}
			;

Vars 			: Var 	{insertInListaVars($1, 1);
				}
			| Vars ',' Var 	{insertInListaVars($3, 0);
					}
			;

Var 			: IDENTIFIER Value_Var { 
							$$.id=$1;
							$$.type=$2.type;
							if ($2.type != -1) {
								$$.value=$2.value;
							}
					       }
			;
	
Value_Var 		: {$$.type=-1;}
			| '=' Inic_Var {$$=$2;}
			;

Type 			: INTEGER {$$ = 0;}
			| BOOLEAN {$$ = 1;}	
			| STRING  {$$ = 2;}
			/*| ARRAY SIZE NUMBER*/
			;
	
Inic_Var 		: Constant {$$ = $1;}
			| '(' Constant ')' {$$ = $2;} 
			/*| Array_Definition*/
			;
	
Constant	 	: NUMBER {$$.value = $1; $$.type=0;}
			| STR 	 {$$.value = $1; $$.type=2;}
			| TRUE   {$$.value = "TRUE"; $$.type=1;}
			| FALSE  {$$.value = "FALSE"; $$.type=1;}
			;
	

/***************************Declarations: Variables: Array_Definition**************/


/*
Array_Definition 	: '[' Array_Initialization ']'
			;
	
Array_Initialization 	: Elem
			| Array_Initialization ',' Elem
			;
	
Elem 			: NUMBER
			;
*/	


/***************************Statements**************/

Statements 		: Statement ';' 
			| Statements Statement ';'
			;
	
Statement 		: Turtle_Commands 
			| Assignment
			| Conditional_Statement
			| Iterative_Statement
			;
	

/***************************Turtle Statement**************/


Turtle_Commands 	: Step
			| Rotate
			| Mode
			| Dialogue
			| Location
			;

Step 			: FORWARD 		{	
							VarData aux2 = searchVar("xpos"), aux3 = searchVar("ypos");
							printf("PUSHG %d\n", aux2->address);	//para o drawline
							printf("PUSHG %d\n", aux3->address);     //para o drawline
					 	}	
			Expression 	{
						VarData aux = NULL;
						if (direccao == up || direccao == down)
							aux = searchVar("ypos");
						else if (direccao == right || direccao == left)
							aux = searchVar("xpos");

						printf("PUSHG %d\n", aux->address);
						printf("SWAP\n"); //para a subtracao ser bem feita

						switch(direccao){
							case(up):
								printf("SUB\n");
								break;
							case(down):
								printf("ADD\n");
								break;
							case(right):
								printf("ADD\n");
								break;
							case(left):
								printf("SUB\n");
								break;
							default:
								break;
						}
						printf("STOREG %d\n", aux->address);
						drawLine();							
						drawTurtle();
					}
			| BACKWARD	{	
							VarData aux2 = searchVar("xpos"), aux3 = searchVar("ypos");
							printf("PUSHG %d\n", aux2->address);	//para o drawline
							printf("PUSHG %d\n", aux3->address);     //para o drawline
					 } 
			 Expression 	 {
						VarData aux = NULL;
			
						if (direccao == up || direccao == down)
							aux = searchVar("ypos");
						else if (direccao == right || direccao == left)
							aux = searchVar("xpos");

						printf("PUSHG %d\n", aux->address);
						printf("SWAP\n"); //para a subtracao ser bem feita

						  switch(direccao){
							  case(up):
								  printf("ADD\n");
								  break;
							  case(down):
								  printf("SUB\n");
								  break;
							  case(right):
								  printf("SUB\n");
								  break;
							  case(left):
								  printf("ADD\n");
								  break;
							  default:
								  break;
						}
						printf("STOREG %d\n", aux->address);
						drawLine();
						drawTurtle();
						}
			;

Rotate 			: RRIGHT				{
								switch(direccao){
									case(up):
										direccao = right;
										break;
									case(right):
										direccao = down;
										break;
									case(down):
										direccao = left;
										break;
									default:
										direccao = up;
										break;
								}
								}
			| RLEFT					{
								switch(direccao){
									case(up):
										direccao = left;
										break;
									case(left):
										direccao = down;
										break;
									case(down):
										direccao = right;
										break;
									default:
										direccao = up;
										break;
								}
								}
			;
	
Mode 			: PEN UP				{ mode = 0; }
			| PEN DOWN				{ mode = 1; }
			;
	
Dialogue 		: Say_Statement
			| Ask_Statement
			;
	
Location 		: GOTO NUMBER ',' NUMBER		{
									VarData aux2 = searchVar("xpos"), aux3 = searchVar("ypos");
									printf("PUSHG %d\n", aux2->address);	//para o drawline
									printf("PUSHG %d\n", aux3->address);	//para o drawline
									VarData aux = searchVar("xpos"), aux1 = searchVar("ypos");
									printf("PUSHI %s\n", $2);
									printf("STOREG %d\n", aux->address);
									printf("PUSHI %s\n", $4);
									printf("STOREG %d\n", aux1->address);
									drawLine();									
									drawTurtle();
									
								}	
			| WHERE	 '?'				{
									VarData aux2 = searchVar("xpos"), aux3 = searchVar("ypos");
									printf("PUSHG %d\n", aux2->address);	//para o drawline
									printf("PUSHG %d\n", aux3->address);     //para o drawline
									aux2 = searchVar("xpos"), searchVar("ypos");
									printf("PUSHG %d\n", aux2->address);
									printf("WRITEI\n");
									printf("PUSHG %d\n", aux3->address);
									printf("WRITEI\n");
								}
			;
	

/***************************Assignment Statement**************/


Assignment 		: Variable '=' Expression 	{
								VarData var =  searchVar($1.id);
								if (var) printf("STOREG %d\n",var->address);
								else yyerror("Variable undeclared!\n");
							}
			;
	
Variable 		: IDENTIFIER 	 		{ 	VarData var = searchVar($1);
								if(var){
									$$.id=var->id;
									$$.type=var->type;
								}
								else $$.type = -1;	
					       		}
			;
/*	
Array_Acess 		:
			| '[' Single_Expression ']'		{ $$ = $2; }
			;
*/	

/***************************Expression**************/

Expression 		: Single_Expression			{ $$ = $1; }
			| Expression Rel_Op Single_Expression	{
									if (strcmp($2,"DIF")!=0)
										printf("%s\n",$2);
									else{	
										printf("EQUAL\n");
										printf("NOT\n");
									}	
								}
			;
	

/***************************Single Expression**************/

Single_Expression 	: Term					{ $$ = $1; }
			| Single_Expression Add_Op Term		{
									if (strcmp($2,"OR")!=0)
									printf("%s\n",$2);
									else{
										printf("ADD\n");
										printf("PUSHI 0\n");
										printf("EQUAL\n");
										printf("NOT\n");
									}
								}
			;


/***************************Term**************/

Term 			: Factor				{ $$ = $1; }
			| Term Mul_Op Factor			{
									if(strcmp($2,"AND")==0){
										printf("MUL\n");
										printf("PUSHI 0\n");
										printf("EQUAL\n");
									        printf("NOT\n");
									}
									else if (strcmp($2,"POW")==0){
										// TODO calcular a potencia
									}
									else printf("%s\n",$2);
								}
			;	


/***************************Factor**************/

Factor 			: Constant				{pushValues($1.type,0,$1.value); $$ = $1.type;}
			| Variable				{
								  VarData var = searchVar ($1.id);
								  if(var)printf("PUSHG %d\n", var->address);
								  else yyerror("Variable undeclared!\n");
								  $$ = var->type;
								}
			| SuccOrPred		{ $$ = $1; }
			| '(' Expression ')'	{ $$ = $2; }
			;
	

/***************************Operators**************/

Add_Op  		: '+'			{ $$ = "ADD"; }
			| '-'			{ $$ = "SUB"; }
			| OR			{ $$ = "OR"; }
			;
	
Mul_Op 			: '*'			{ $$ = "MUL"; }
			| '/'			{ $$ = "DIV"; }
			| AND			{ $$ = "AND"; }
			| POW			{ $$ = "POW"; }
			;
	 
Rel_Op 			: EQUAL			{ $$ = "EQUAL"; }
			| DIF 			{ $$ = "DIF"; }
			| MAJOR 		{ $$ = "SUP"; }
			| MINOR			{ $$ = "INF"; }
			| MAJOREQUAL		{ $$ = "SUPEQ"; }
			| MINOREQUAL		{ $$ = "INFEQ"; }
			/*| IN			{ $$ = $1; }*/       //PENSO QUE SEJA SO PARA ARRAYS
			;
	

/***************************SuccOrPred**************/

SuccOrPred 		: SuccPred IDENTIFIER		{ VarData var = searchVar($2);
							  if (var) {
							  	  printf("PUSHG %d\n", var->address);
								  printf("PUSHI %d\n", $1);
								  printf("ADD\n");
							  }
							  else yyerror("Variable undeclared!\n");
							}
			;
	
SuccPred 		: SUCC				{ $$ = 1; }
			| PRED				{ $$ = -1; }
			;


/***************************IO Statements***********/
	
Say_Statement 		: SAY '(' Expression ')'		{ switch ($3){ // Expression Type 
									case 0:	// INTEGER							
										printf("WRITEI\n"); 
										break;
									case 1: // BOOLEAN
										printf("WRITEI\n");
										break;
									case 2: // STRING
										printf("WRITES\n");
										break;
									}
								}
			;						
			
	
Ask_Statement 		: ASK '(' STR ',' Variable ')'		{ 
								  VarData var = searchVar($5.id);
								  if(!var) yyerror("Variable undeclared!\n");
								  else{
									printf("PUSHS %s\n",$3); 	// guardar na stack a STR a perguntar
								  	printf("WRITES\n"); 		// escrever a STR a perguntar
									printf("READ\n"); 	/* lê uma string do teclado (concluída por um "\n") 
										       		   e arquiva esta string (sem o "\n") na heap e coloca
                                                                                       		   (empilha) o endereço na pilha..
									            		*/
									switch ($5.type){ // Expression Type
										case 0:	// INTEGER							
											printf("atoi\n"); 
											break;
										case 1: // BOOLEAN
											printf("atoi\n");
											break;
										case 2: // STRING
											break;
										default :
											yyerror("Variable undeclared!\n");
											break;
									  }
									  printf("STOREG %d\n",var->address);
									}
								  }
			;
	

/***************************Consitional & Iterative Statements*******/

Conditional_Statement	: IfThenElse_Stat
			;
	
Iterative_Statement 	: While_Stat
			;
	

/***************************IfThenElse_Stat*********/

IfThenElse_Stat 	: IF Expression {}
			  THEN '{' Statements '}' Else_Expression
			;	

Else_Expression 	:
			| ELSE '{' Statements '}'
			;
	

/***************************While_Stat**************/

While_Stat 		: WHILE '(' Expression ')' '{' Statements '}'
			;

%%

void insertInListaVars(VarTipo var, int first){
	ListaVars *aux = (ListaVars*)malloc(sizeof(ListaVars));
	aux->id = var.id;
	aux->value = var.value;
	aux->type = var.type;
	if(first == 1){aux->next = NULL;}
	else {aux->next = nodo;}
	nodo = aux;
	
}

void saveVars(int type){
	ListaVars *aux = nodo;
	while(aux) {
		if(!searchVar(aux->id)){
			// insere nome, tipo e address na hashtable
			insertVar(aux->id, type, addressG);
			pushValues(type,aux->type, aux->value);
			addressG++;
		}
		aux=aux->next;
	}
	nodo = NULL;

}

void pushValues(int varType, int nullType, char* value){
	switch(varType) {
		case 0://INTEGER
			if (nullType == -1) //VAZIO
				printf("PUSHI 0\n");
			else 
				printf("PUSHI %d\n",atoi(value));
		break;
		case 1://BOOLEAN
			if (nullType==-1 || strcmp(value,"TRUE")==0) 
				printf("PUSHI 1\n");
			else if (strcmp(value, "FALSE")==0) 
				printf("PUSHI 0\n");
		break;
		case 2://STRING
			if (nullType == -1) 
				printf("pushs \"\"\n");
			else 
				printf("pushs %s\n",value);
		break;
		// nao estamos a fazer arrays para ja
	}
}

void drawTurtle(){
	VarData aux, aux2, aux3;
	aux3 = searchVar("xpos");
        printf("PUSHG %d\n", aux3->address);
	aux2 = searchVar("ypos");
        printf("PUSHG %d\n", aux2->address);
	aux = searchVar("raio");
	printf("PUSHG %d\n", aux->address);
	printf("DRAWCIRCLE\n");
	printf("REFRESH\n");
}

void drawLine(){
	printf("CLEARDRAWINGAREA\n");
	if(mode == 1){ // PEN DOWN
		VarData aux1, aux2;
		aux1 = searchVar("xpos");
        	printf("PUSHG %d\n", aux1->address);
		aux2 = searchVar("ypos");
        	printf("PUSHG %d\n", aux2->address);
		printf("DRAWLINE\n");
	}
}

void init() {
	varHashTable = initHash();
	VarTipo var;

	var.id = "xpos";
	var.value = (char*)malloc(sizeof(10));
	sprintf(var.value, "%d", xpos);
	insertInListaVars(var, 1);

	var.id = "ypos";
	var.value = (char*)malloc(sizeof(10));
	sprintf(var.value, "%d", ypos);
	insertInListaVars(var, 0);

	var.id = "raio";
	var.value = (char*)malloc(sizeof(10));
	sprintf(var.value, "%d", raio);
	insertInListaVars(var, 0);
	
	saveVars(0);
	printf("START\n");
	initWindow();
	drawTurtle();
}

void initWindow(){
	printf("PUSHI %d\n",800);
        printf("PUSHI %d\n",600);
        printf("opendrawingarea\n");
}


int yyerror(char *s){
       fprintf(stderr,"ERRO: %s na linha:%d antes de:%s\n",s,yylineno,yytext);
       return 0;
}

int main() {
	yyparse();
	return 0;
}



++++++++++++++++++++   structures.h  ++++++++++++++++

#ifndef _STRUCTS
#define _STRUCTS

typedef enum Direccoes{
	up,
	down,
	right,
	left
} Direccao;

typedef struct VarTipos {
	char* id;
	char* value;
	int type;
} VarTipo;

typedef struct ConstTipos {
	char* value;
	int type;
} ConstTipo;

typedef struct NodoVar {
	char* id;
	char* value;
	int type;
	struct NodoVar *next;
} ListaVars;


int height, width, xpos, ypos, raio, mode;
Direccao direccao;
ListaVars *nodo;

void insertInListaVars(VarTipo var, int first);
void saveVars(int type);
void drawTurtle();
void drawLine();
void init();
void initWindow();
void pushValues(int varType, int nullType, char* value);

#endif







++++++++++++++++  hashFunctions.c   ++++++++++++++++++++++++

#include "hashFunctions.h"
#include <stdlib.h>
#include <string.h>
#include <stdio.h>


VarHashTable varHashTable;

VarHashTable initHash(){
	int i;
	VarHashTable v;
	v = (VarHashTable)malloc(sizeof(VarData)*HASH_SIZE);
	for (i = 0; i < HASH_SIZE; i++)
		v[i]=NULL;
	return v;
}

int hash(char* s) {
	int h = 0, i, n=strlen(s);
	for (i = 0; i < n; i++) {
		h = 31*h + s[i];
	}
	return abs(h%HASH_SIZE);
}

VarData searchVar(char* id){
	int h = hash(id);
	VarData varData = varHashTable[h];
	while (varData && strcmp(varData->id, id)!=0){
		varData = varData->next;
	}
	return varData;
}

int insertVar(char* id, int type, int address) {
	int h = hash(id);
	VarData new, var;
	new = (VarData)malloc(sizeof(struct varData));
	var = varHashTable[h];
	varHashTable[h]=new;
	new->id=id;
	new->type=type;
	new->address=address;
	new->next=var;
	return 1;
}








++++++++++++++++  hashFunctions.h   +++++++++++++++++++++++

#ifndef _HASH
#define _HASH

#define HASH_SIZE 30

typedef struct varData {
	char* id;
	int type;
	int address;
	struct varData *next;
} *VarData, **VarHashTable;

VarHashTable initHash();
int hash(char* s);
VarData searchVar(char* id);
int insertVar(char* id, int type, int address);
void printHash();

#endif







+++++++++++++++++  Makefile   +++++++++++++++++++++++++

logoliss: y.tab.o lex.yy.o hashFunctions.o structures.h
	cc -o logoliss y.tab.o lex.yy.o hashFunctions.o -lfl

y.tab.o: y.tab.c
	cc -c y.tab.c

lex.yy.o: lex.yy.c
	cc -c lex.yy.c


lex.yy.c: logo.l y.tab.h structures.h
	flex logo.l

y.tab.c y.tab.h: logo.y
	yacc -d -v logo.y

hashFunctions.o: hashFunctions.h hashFunctions.c
	cc -c hashFunctions.c

clean : 
	rm -Rf *.o y.* lex.yy.c
\end{verbatim}

\end{codefold}
\end{document}
